\chapter{Conclusion}
\thispagestyle{chapterstart}

The transition to post-quantum cryptography is imperative in the face of advancing quantum computing capabilities that threaten the security of classical cryptographic schemes. However, implementing post-quantum algorithms like CRYSTALS-Dilithium securely on embedded devices presents significant challenges due to side-channel vulnerabilities inherent in practical implementations.

In this paper, we analyzed the critical aspects affecting the feasibility and performance of side-channel resistant implementations of post-quantum schemes, focusing on Dilithium. We identified key vulnerabilities, such as those in the bit-unpacking function and the Number Theoretic Transform (NTT), which can be exploited to recover secret keys through side-channel attacks. To address these vulnerabilities, we examined various countermeasures, including initial masking techniques, subsequent improvements with optimized masking gadgets, and shuffling methods.

Through a comparative analysis, we visualized and discussed existing implementations regarding their security and performance on embedded devices. Our findings indicate that while higher-order masking techniques offer strong protection against side-channel attacks, they often introduce substantial performance overheads, making them less feasible for resource-constrained environments. Conversely, shuffling techniques and optimized masking methods can provide a more favorable balance between security and efficiency.

The trade-offs between security and performance are critical when selecting appropriate countermeasures for PQC implementations on embedded devices. Practitioners must consider the specific security requirements and constraints of their applications. Our study highlights the importance of continued research and development in optimizing side-channel countermeasures, exploring hybrid approaches that combine multiple techniques, and tailoring implementations to the capabilities of target devices.

In conclusion, enhancing the side-channel resistance of post-quantum cryptographic schemes is essential for their secure deployment in the quantum era. By carefully selecting and optimizing countermeasures, it is possible to achieve practical and secure PQC implementations on embedded systems. Future work should focus on further reducing the performance impact of countermeasures, developing standardized methodologies for side-channel resistance, and ensuring that security evaluations keep pace with evolving attack strategies.
