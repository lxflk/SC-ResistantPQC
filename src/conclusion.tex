\chapter{Conclusion}
\thispagestyle{chapterstart}

The secure deployment of post-quantum cryptographic (PQC) schemes like CRYSTALS-Dilithium on embedded devices is essential in the face of advancing quantum computing capabilities. However, practical implementations are susceptible to side-channel attacks due to inherent vulnerabilities, such as those in the bit-unpacking function and the Number Theoretic Transform (NTT).

In this paper, we analyzed the feasibility and performance of side-channel resistant implementations of Dilithium on embedded devices. We identified key vulnerabilities that can be exploited to recover secret keys and examined various countermeasures, including masking techniques and shuffling methods.

Our comparative analysis showed that high-order masking offers strong protection against side-channel attacks under rigorous security models like the $t$-probing model. However, such techniques often incur substantial performance overheads, particularly in non-optimized implementations. Optimizing these implementations through architecture-specific techniques, such as bitslicing, can mitigate performance penalties and make them more practical for resource-constrained environments.

Shuffling techniques and optimized masking methods provide a more favorable balance between security and efficiency. Shuffling introduces randomness to operation sequences, reducing predictability and side-channel leakage with lower performance impact. Optimized masking schemes, as demonstrated by Azouaoui et al. \cite{Azouaoui22}, enhance performance by focusing on critical operations and employing efficient gadgets.

Selecting appropriate countermeasures requires careful consideration of security requirements, performance constraints, and hardware characteristics. Practitioners should balance the desired level of side-channel resistance with acceptable performance overheads, tailoring implementations to their specific applications.

Future work should focus on developing hybrid approaches that combine masking with shuffling or bitslicing to achieve both high security and efficiency. Further optimization of implementations for specific architectures, mitigation of micro-architectural leakages, and exploration of alternative security models can enhance the practicality of side-channel resistant PQC schemes.

In conclusion, enhancing side-channel resistance is crucial for the secure adoption of PQC schemes in the quantum era. By carefully selecting and optimizing countermeasures, it is possible to implement practical and secure Dilithium on embedded devices. Ongoing research and development are essential to address the challenges of side-channel attacks and to ensure the robustness of cryptographic implementations against evolving threats.
