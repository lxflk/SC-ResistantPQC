% conclusion.tex

\chapter{Conclusion}
\thispagestyle{chapterstart}

In this paper, we analyzed and compared several side-channel resistant implementations of the CRYSTALS-Dilithium digital signature scheme, focusing on the most important aspects affecting feasibility and performance on embedded devices. Our evaluation centered on security levels achieved (masking order), performance metrics, implementation techniques, and the trade-offs between security and efficiency.

Our comparative analysis revealed that while all implementations aim to enhance side-channel resistance, they differ in their approaches and trade-offs. Migliore et al.\ \cite{Migliore19} introduced a first-order masked implementation using a power-of-two modulus to simplify masking and improve efficiency. However, modifying the original Dilithium scheme may affect standardization and security proofs. Azouaoui et al.\ \cite{Azouaoui22} provided a refined sensitivity analysis and developed optimized masking gadgets tailored for Dilithium, supporting higher-order masking up to $d=8$ without altering the scheme, making their implementation suitable for practical deployment on embedded devices. Coron et al.\ \cite{Coron23} proposed innovative masking gadgets like the ShiftMod gadget, achieving efficiency improvements at small masking orders, but facing scalability limitations due to exponential complexity at higher orders.

In terms of feasibility and performance, Azouaoui et al.'s implementation offers the best balance, providing strong side-channel resistance with reasonable performance on embedded devices, and maintaining compliance with the standard Dilithium parameters. Their use of optimized gadgets and randomized signing reduces computational overhead and simplifies the masking of complex operations.

Our findings highlight that achieving side-channel resistance in post-quantum cryptographic schemes requires careful consideration of masking techniques and implementation strategies. Key aspects affecting feasibility and performance include the masking order, efficiency of masking gadgets, compliance with standard parameters, and the specific characteristics of the target embedded devices. By focusing on these aspects, practitioners can balance security and performance to deploy side-channel resistant Dilithium implementations effectively in real-world applications.

Enhancing the side-channel resistance of post-quantum schemes like Dilithium is crucial for secure deployments in the quantum era. Future work should focus on further optimizing masking techniques to reduce computational overhead, improving scalability for higher-order masking, and exploring additional countermeasures to strengthen security against side-channel attacks. By refining these implementations, we can ensure that post-quantum cryptographic schemes remain robust and practical for widespread adoption in resource-constrained environments.
