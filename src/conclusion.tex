\chapter{Conclusion}
\thispagestyle{chapterstart}

\section{Summary of Findings}

This paper has provided an extensive examination of side-channel resistant implementations of the CRYSTALS-Dilithium signature scheme on embedded devices. We began by highlighting the imminent transition towards post-quantum cryptographic (PQC) standards, driven by the rising threat of quantum computing. Within this context, ensuring that PQC schemes—like Dilithium—can be securely and efficiently deployed on resource-constrained hardware is both timely and imperative.

Our analysis revealed that Dilithium, despite its theoretical resilience against quantum adversaries, is susceptible to a range of side-channel attacks when implemented on embedded platforms. Notably, functions such as the bit-unpacking process and the Number Theoretic Transform (NTT) exhibit intrinsic vulnerabilities that allow attackers to infer secret information by analyzing power consumption, electromagnetic emissions, or other physical signals. This underscores the necessity of implementing robust countermeasures to mitigate such leakages.

We surveyed and evaluated several protection strategies, with particular attention to masking and shuffling techniques. While higher-order masking schemes demonstrate substantial resistance to side-channel analysis, they also introduce significant computational overhead, which can be prohibitive in real-world embedded environments. In contrast, shuffling approaches, or suitably optimized masking frameworks, often present a more balanced compromise between security and performance. Through the use of refined sensitivity analyses, bitsliced implementations, and the design of innovative masking gadgets, recent research has shown that performance penalties can be mitigated without sacrificing the robustness of the underlying security guarantees.

\section{Implications for Practice}

For practitioners and system designers, our comparative analysis offers clear guidance. Decisions about which countermeasures to employ must account for the security requirements of the target application, available computational resources, and acceptable performance trade-offs. While no single technique emerges as a universally optimal solution, combining multiple approaches—such as layering shuffling techniques over a carefully selected masking scheme—can yield robust, context-sensitive defenses against side-channel adversaries.

Moreover, this research emphasizes the importance of continuous refinement and verification. It is not sufficient to rely solely on theoretical models or isolated security proofs; rigorous experimental evaluations on representative embedded platforms are essential. Such an empirical approach ensures that solutions remain effective in practice, rather than merely on paper.

\section{Directions for Future Research}

The ongoing standardization efforts by NIST and others will guide the adoption of PQC schemes in the coming years. As these algorithms become entrenched in practical applications, the need for increasingly sophisticated side-channel protections will only intensify. Future work should focus on:

\begin{itemize}
    \item \textbf{Hybrid Countermeasures}: Exploring new combinations of masking, shuffling, bitslicing, and other techniques to reduce overhead while retaining strong security guarantees.
    \item \textbf{Architecture-Specific Optimizations}: Tailoring implementations to the microarchitectural features of emerging embedded systems can enhance both performance and security, ensuring that countermeasures are well-aligned with the device’s capabilities.
    \item \textbf{Extended Attack Models}: Addressing micro-architectural and fault-injection attacks, which extend beyond traditional side-channel leakage, to create even more resilient implementations.
\end{itemize}

\section{Final Remarks}

The transition to the quantum era calls for cryptographic solutions that remain secure under new, more powerful adversarial models. While Dilithium and other PQC schemes offer theoretical security against quantum attacks, their safe deployment on embedded devices depends critically on effective, pragmatic side-channel resistance. This paper’s findings underscore that although challenges remain—particularly concerning performance overheads—ongoing research, improved methodologies, and careful engineering hold the promise of surmounting these obstacles.

In conclusion, the secure, efficient implementation of Dilithium on embedded platforms is achievable with informed strategy, rigorous optimization, and a commitment to addressing emerging threats. Such efforts will be integral to ensuring the reliability and trustworthiness of post-quantum cryptography in the decades to come.
