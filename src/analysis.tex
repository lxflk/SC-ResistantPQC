% analysis.tex

\chapter{Analysis}
\thispagestyle{chapterstart}

\section{Methodology}

In this chapter, we evaluate the side-channel resistance and performance of various masked implementations of the CRYSTALS-Dilithium digital signature scheme. Our methodology focuses on comparing key metrics across selected implementations to understand the trade-offs between security and efficiency, particularly in the context of embedded devices.

The main aspects considered are:

\begin{itemize}
    \item \textbf{Security Level}: The order of masking applied and robustness against side-channel attacks.
    \item \textbf{Performance Metrics}: Execution time, cycle counts, and computational overhead introduced by the masking techniques.
    \item \textbf{Feasibility on Embedded Devices}: Practicality of implementations on resource-constrained platforms.
    \item \textbf{Implementation Techniques}: Specific masking schemes and optimizations employed to enhance security and performance.
    \item \textbf{Trade-offs}: Balancing security requirements with performance constraints.
\end{itemize}

We base our analysis on key papers that have significantly contributed to side-channel resistant implementations of Dilithium.

\section{Related Work}

\subsection{Masking Dilithium: Efficient Implementation and Side-Channel Evaluation}

Migliore et al.\ \cite{Migliore19} introduce a first-order masked implementation of Dilithium, focusing on efficient techniques suitable for embedded devices. They address the challenges of masking lattice-based operations and aim to minimize the performance overhead associated with side-channel countermeasures. The authors propose using a power-of-two modulus to simplify masking and improve efficiency without affecting security.

\subsubsection{Security Level}

The implementation targets first-order side-channel resistance by applying masking to sensitive variables. The authors identify the parts of the algorithm requiring protection, such as secret keys and intermediate computations involving these keys.

\subsubsection{Performance Metrics}

Key performance metrics reported include:

\begin{itemize}
    \item \textbf{Execution Time}: The masked implementation introduces an overhead of approximately $5.6\times$ compared to the unmasked version.
    \item \textbf{Cycle Counts}: Optimizations, including the use of a power-of-two modulus, reduce cycle counts for critical operations, improving performance.
\end{itemize}

\subsubsection{Feasibility on Embedded Devices}

The authors demonstrate that their first-order masked implementation is practical on embedded platforms. By using a power-of-two modulus, they achieve acceptable performance suitable for resource-constrained devices.

\subsubsection{Implementation Techniques}

They employ specialized algorithms for masking polynomial arithmetic and optimize the implementation by minimizing non-linear operations that require costly masking. The key innovation is the use of a power-of-two modulus, simplifying the masking of decomposition functions and reducing complexity.

\subsection{Protecting Dilithium Against Leakage Revisited}

Azouaoui et al.\ \cite{Azouaoui22} revisit the masking of Dilithium by performing a refined sensitivity analysis and introducing improved masking gadgets tailored to Dilithium. They identify flaws in previous analyses, such as that of Migliore et al., and propose corrections leading to more secure and efficient implementations.

\subsubsection{Security Level}

Their implementations support higher-order masking, providing security against side-channel attacks up to the eighth masking order ($d=8$). By accurately classifying the sensitivity of intermediate computations, they ensure that only necessary components are masked, avoiding insecurity due to unprotected variables and inefficiency from unnecessary masking.

\subsubsection{Performance Metrics}

Performance improvements are achieved through:

\begin{itemize}
    \item \textbf{Optimized Masking Gadgets}: Introduction of new gadgets specifically designed for Dilithium operations, leveraging advances in masking conversion algorithms.
    \item \textbf{Randomized vs.\ Deterministic Signing}: Demonstrating that randomized signing can lead to significantly more efficient implementations when side-channel attacks are a concern.
\end{itemize}

For first-order masking ($d=2$), their randomized implementation requires approximately 13.9 million cycles on an ARM Cortex-M4 microcontroller, significantly faster than the deterministic version.

\subsubsection{Feasibility on Embedded Devices}

The implementations are practical on embedded devices for various masking orders. The authors provide benchmarks on an ARM Cortex-M4, showing acceptable performance even at higher masking orders due to their optimized gadgets.

\subsubsection{Implementation Techniques}

Key strategies include:

\begin{itemize}
    \item \textbf{Refined Sensitivity Analysis}: Re-examining the sensitivity of variables and operations, correcting previous misconceptions, and ensuring proper protection of sensitive variables.
    \item \textbf{Improved Masking Gadgets}: Developing new gadgets dedicated to Dilithium's operations, such as bound checks and the decomposition function.
    \item \textbf{PINI-Compliant Gadgets}: Utilizing \ac{PINI} compliant gadgets to ensure security and enable secure composition of masked operations.
    \item \textbf{Randomized Signing}: Exploiting randomized signing over deterministic signing to achieve better performance and security against side-channel attacks.
\end{itemize}

\subsection{Improved Gadgets for the High-Order Masking of Dilithium}

Coron et al.\ \cite{Coron23} introduce a set of novel masking gadgets designed to improve the efficiency of high-order masking in the Dilithium signature scheme. Their work addresses key challenges in masking complex operations inherent in lattice-based cryptography, particularly focusing on reducing the computational overhead associated with high-order masking while ensuring robust security guarantees.

\subsubsection{Security Level}

The proposed gadgets are tailored to achieve security against high-order side-channel attacks. Coron et al.\ provide formal security proofs in the $t$-probing model, which is a standard framework for analyzing the security of masked implementations against side-channel attacks. In this model, an adversary can probe up to $t$ intermediate variables during the computation, and the implementation is considered secure if these probes do not reveal any sensitive information about the secret keys.

Their gadgets are designed to be secure for any order $t$, with a particular focus on practical values of $t$ relevant for real-world applications. The formal proofs ensure that the masked operations do not introduce vulnerabilities that could be exploited by sophisticated attackers capable of higher-order attacks.

\subsubsection{Performance Metrics}

A significant contribution of their work is the introduction of the \textbf{ShiftMod} gadget, which efficiently performs arithmetic shifts modulo $2q$. This gadget reduces the complexity of such operations, which are frequently used in the polynomial arithmetic of Dilithium.

Specifically, the ShiftMod gadget allows for:

\begin{itemize}
    \item \textbf{Lower Computational Complexity}: By optimizing the arithmetic shift operations, they reduce the number of required operations, leading to faster execution times.
    \item \textbf{Scalability for Small Orders}: The performance improvements are particularly notable for small masking orders (e.g., $d \leq 6$), where the overhead is significantly reduced compared to previous implementations.
\end{itemize}

Additionally, they improve the \textbf{Boolean-to-Arithmetic Conversion} process, which is essential in masked implementations where values may need to be converted between different masking domains. Their method achieves lower operation counts for small $d$, making the overall implementation more efficient.

However, they acknowledge that the complexity of their approach increases exponentially with the masking order $d$. As a result, while the gadgets provide substantial performance benefits for lower orders, the practicality diminishes for higher orders due to the exponential growth in computational requirements.

\subsubsection{Feasibility on Embedded Devices}

For small masking orders, their implementation is feasible on embedded devices, offering improved performance without compromising security. The efficiency gains from the ShiftMod gadget and optimized conversion algorithms make the masked Dilithium implementation suitable for resource-constrained environments when high-order masking is not required.

However, due to the exponential increase in complexity with higher masking orders, the implementation becomes less practical for embedded devices when $d$ is large. The computational and memory requirements may exceed the capabilities of typical embedded platforms, limiting the applicability of their approach in scenarios where very high security levels are mandated.

\subsubsection{Implementation Techniques}

Coron et al.\ employ several innovative techniques to enhance the efficiency of the masked implementation:

\begin{itemize}
    \item \textbf{ShiftMod Gadget}: This new gadget efficiently computes arithmetic shifts modulo $2q$ by leveraging properties of modular arithmetic and masking schemes. It reduces the number of required operations for shift computations, which are common in polynomial arithmetic and in functions like \texttt{Decompose}.
    \item \textbf{Optimized Masking of \texttt{Decompose} Function}: They provide improved methods for masking the \texttt{Decompose} function, a critical component in Dilithium that splits polynomial coefficients into higher and lower bits. Their approach reduces the complexity of this operation, leading to better performance in the masked implementation.
    \item \textbf{Efficient Conversion Algorithms}: The authors introduce optimized algorithms for converting between Boolean and arithmetic masking representations. These conversions are necessary in masked implementations to securely perform different types of operations, and their efficient algorithms reduce the overhead associated with these conversions.
    \item \textbf{Security Proofs in the Probing Model}: By providing formal security proofs, they ensure that their masking techniques are sound and that the implementation is secure against attacks modeled in the $t$-probing framework. This adds confidence in the robustness of their approach.
\end{itemize}

Their work builds upon previous research in high-order masking but focuses on tailoring the gadgets specifically for the operations used in Dilithium. By addressing the unique challenges posed by lattice-based cryptography, they contribute valuable tools for enhancing the side-channel resistance of post-quantum cryptographic schemes.

