\chapter{Analysis}
\thispagestyle{chapterstart}

In this chapter, we analyze the side-channel vulnerabilities inherent in the Dilithium lattice-based signature scheme and explore the countermeasures developed to mitigate these weaknesses. Our focus is on the feasibility and performance of side-channel resistant implementations on embedded devices, emphasizing the practical implications of these vulnerabilities and the effectiveness of the proposed countermeasures.

\section{Overview}

The structure of this analysis is as follows:

\begin{itemize}
    \item \textbf{Side-Channel Vulnerabilities in Dilithium}: We examine specific vulnerabilities in the bit-unpacking function and the \ac{NTT} used during signature generation, highlighting how they can be exploited to recover secret keys.
    \item \textbf{Initial Masking Techniques as Countermeasures}: We explore the initial masking techniques proposed to protect Dilithium against side-channel attacks, discussing their implementation and effectiveness.
    \item \textbf{Subsequent Improvements}: We outline advancements made in masking techniques to enhance security and performance on embedded devices.
    \item \textbf{Shuffling Techniques as Alternative Countermeasures}: We introduce shuffling as an alternative or complementary countermeasure, detailing its potential to mitigate side-channel vulnerabilities, particularly in the \ac{NTT}.
\end{itemize}

\section{Side-Channel Vulnerabilities in Dilithium}

Dilithium implementations are susceptible to side-channel attacks due to certain operations that leak sensitive information through physical channels like power consumption or electromagnetic emissions. Two significant vulnerabilities have been identified:

\subsection{Leakage in the Bit-Unpacking Function}

One critical vulnerability arises from the bit-unpacking function used during signature generation. This function is responsible for converting a byte buffer into polynomial coefficients, specifically the random vector $y$, which is crucial in the computation of the signature.

Marzougui et al.~\cite{Marzougui22} identified that the bit-unpacking function leaks information due to differences in power consumption patterns when processing zero and non-zero coefficients. The unpacking process involves bitwise operations that vary depending on the value of the input bits. When a coefficient is zero, the function executes a different sequence of operations compared to when the coefficient is non-zero. These differences manifest as distinguishable patterns in power traces or electromagnetic emissions.

By capturing power traces during the execution of the bit-unpacking function, an attacker can apply statistical methods and machine learning techniques to classify the traces and determine whether each coefficient of $y$ is zero or non-zero. This classification can be done with high accuracy, even in the presence of noise, by training classifiers on a set of labeled traces obtained from a similar device under the attacker's control.

\subsection{Leakage in the Number Theoretic Transform (NTT)}

Another significant vulnerability is associated with the \ac{NTT}, a critical component in Dilithium for efficient polynomial multiplication. The \ac{NTT} implementation can introduce side-channel leakage due to its deterministic execution patterns and data dependencies.

Primas et al.~\cite{Primas17} and Pessl et al.~\cite{Pessl19} demonstrated that side-channel attacks targeting the \ac{NTT} can effectively compromise the security of lattice-based schemes like Dilithium. By analyzing power consumption during the execution of the \ac{NTT}, attackers can extract information about the secret polynomials used in the scheme.

Specifically, these attacks exploit the fact that the \ac{NTT} involves arithmetic operations whose intermediate values depend on the secret data. The execution of butterfly operations within the \ac{NTT} can produce distinguishable patterns in power traces. By collecting a single trace and applying sophisticated analytical techniques, attackers can recover the secret key.

\subsection{Exploitation for Secret Key Recovery}

The leakage from both the bit-unpacking function and the \ac{NTT} becomes critical when considering how it can be exploited to recover the secret key component $s_1$.

\subsubsection{Exploiting Bit-Unpacking Function Leakage}

In Dilithium, the signature is computed as $z = y + c s_1$, where $c$ is a known challenge value derived from the message and public parameters. Since $z$ and $c$ are known to the attacker, knowledge of $y$ allows the attacker to solve for $s_1$:

\[
    s_1 = \frac{z - y}{c}
\]

By accurately determining which coefficients of $y$ are zero, the attacker gains partial information about $s_1$. Repeating this process over multiple signatures and leveraging linear algebra techniques, the attacker can construct a system of equations to solve for $s_1$ completely. Marzougui et al.~\cite{Marzougui22} demonstrated that with sufficient power traces and appropriate analytical methods, such as integer linear programming and least squares estimation, it is possible to recover $s_1$ and thus compromise the security of the scheme.

\subsubsection{Exploiting \ac{NTT} Leakage}

The \ac{NTT}-related attacks focus on recovering secret polynomials used in the scheme by analyzing side-channel emissions during the \ac{NTT} computation. Since the \ac{NTT} is used to perform polynomial multiplication with secret inputs, leakage during its execution can reveal information about these secrets.

Primas et al.~\cite{Primas17} utilized a single-trace attack where they modeled the \ac{NTT} as a factor graph and applied belief propagation algorithms to infer the secret data from side-channel observations. Pessl et al.~\cite{Pessl19} improved upon this by reducing the number of templates required and demonstrating the feasibility of the attack on embedded devices like the ARM Cortex-M4.

By targeting specific operations within the \ac{NTT}, such as the butterfly computations, and exploiting the known structure and deterministic execution, attackers can reconstruct the secret polynomials with high accuracy.

\section{Initial Masking Techniques as Countermeasures}

To mitigate side-channel vulnerabilities in Dilithium, masking techniques split sensitive variables into multiple random shares, so each share individually reveals no information about the secret. This approach aims to decorrelate side-channel leakage from sensitive data.

Migliore et al.~\cite{Migliore19} presented an initial masking of Dilithium to protect against first-order side-channel attacks. They first identify sensitive operations, including the secret polynomials $s_1$ and $s_2$, the random vector $y$, and certain intermediate variables.

Arithmetic masking is applied to linear operations involving modular arithmetic, such as sampling $s_1$ and $s_2$ and polynomial multiplications. For non-linear operations and the random vector $y$, Boolean masking is used. This dual approach requires secure conversions between arithmetic and Boolean domains, implemented via algorithms that prevent leakage during conversion.

They designed specific masked gadgets for critical functions like polynomial multiplication, rejection sampling, and the decomposition function \texttt{Decompose$_q$}, ensuring security under the probing model. Generating high-quality randomness for shares and mask refreshing is essential for maintaining security.

Overall, the initial masking techniques by Migliore et al.\cite{Migliore19} provide a foundation for protecting Dilithium against side-channel attacks. By carefully analyzing sensitive components and designing secure masked implementations, they demonstrate that achieving side-channel resistance in practical implementations of Dilithium is feasible, despite the inherent challenges.



\section{Subsequent Improvements}

Following the initial masking techniques proposed to protect Dilithium against side-channel attacks, further research has focused on enhancing both the security and performance of masked implementations. Two significant contributions in this area are the works by Azouaoui et al.~\cite{Azouaoui22} and Coron et al.~\cite{Coron23}, which introduce refined sensitivity analyses and improved masking gadgets tailored for Dilithium.

\subsection{Revisited Sensitivity Analysis and Improved Implementations}

Azouaoui et al.~\cite{Azouaoui22} revisited the sensitivity analysis of Dilithium's key generation and signing procedures to accurately determine which intermediate computations and variables require protection against side-channel leakage. Their analysis identified discrepancies in previous work by Migliore et al.~\cite{Migliore19}, where some public or non-sensitive variables were unnecessarily protected, and certain sensitive variables were left unprotected.

They emphasized the need to protect secret key components $s_1$ and $s_2$, as well as intermediate variables such as the random vector $y$ and the polynomial $w$ computed during signature generation. By clarifying that variables like the public key components, the challenge $c$, and the hint vector $h$ do not require side-channel protection after specific checks have passed, they enabled performance optimizations. Additionally, they identified that some variables previously considered sensitive, such as $w_1$ and $\tilde{r}$, can be safely left unprotected under certain conditions based on the zero-knowledge proofs and security reductions in Dilithium.

Building on this refined sensitivity analysis, they developed improved masked implementations of Dilithium that leverage specialized masking gadgets. They introduced optimized gadgets tailored for Dilithium's specific operations, such as the \texttt{SecLeq}, \texttt{SecBoundCheck}, \texttt{SecSampleModp}, and \texttt{SecDecompose} functions. These gadgets are designed to be efficient and secure under the probing security model, ensuring that any adversary observing up to $d-1$ shares learns no information about the secret variables.

\subsection{Novel High-Order Masking Gadgets}

Coron et al.~\cite{Coron23} advanced side-channel countermeasures by introducing new high-order masking gadgets specifically designed for Dilithium. Their work addresses both security and efficiency challenges in implementing masked versions of Dilithium on embedded devices.

A key contribution of their research is the development of the \texttt{ShiftMod} gadget, which enables efficient arithmetic shifts modulo any integer. This gadget is crucial for operations like arithmetic-to-Boolean masking conversion and improves the masking of the \texttt{Decompose} function in Dilithium. They also proposed a new algorithm for converting integers from Boolean to arithmetic masking modulo any integer $q$, with complexity independent of the bit size of the integer or modulus. This algorithm is particularly effective for generating the random vector $y$ in Dilithium. Moreover, they introduced enhanced techniques for masking the \texttt{Decompose} function, applicable across all security levels of Dilithium, offering better performance compared to previous approaches.

Their implementation results demonstrated significant efficiency gains, with substantial reductions in execution times for critical operations, making high-order masked implementations of Dilithium more practical on embedded devices. They provided formal proofs of security for their gadgets in the $t$-probing model, ensuring robustness against high-order side-channel attacks. Additionally, they showed that their gadgets can be integrated into existing implementations, enhancing both security and performance without requiring substantial changes to the Dilithium algorithm.


\section{Shuffling Techniques as Alternative Countermeasures}

In addition to masking, shuffling techniques have been proposed as alternative or complementary countermeasures against side-channel attacks on the \ac{NTT} in Dilithium. Shuffling randomizes the order of operations to prevent attackers from correlating side-channel observations with specific computations, thus disrupting deterministic execution patterns.

Ravi et al.~\cite{Ravi20} introduced shuffling countermeasures specifically targeting the \ac{NTT} to protect against single-trace side-channel attacks. Their methods include both coarse and fine shuffling techniques.


\subsubsection{Coarse Shuffling}

In Coarse-Full Shuffling, all butterfly operations within each stage of the \ac{NTT} are executed in a random order. This is achieved by generating a random permutation of the operations using algorithms like the Knuth-Yates shuffle. The entropy introduced by this method makes it infeasible for attackers to align side-channel observations with specific operations.

Coarse In-Group Shuffling reduces the overhead by limiting the randomization to groups of butterflies that use the same twiddle factors. This reduces the amount of randomness needed and improves performance while still providing substantial protection.

\subsubsection{Fine Shuffling}

Basic Fine Shuffling focuses on randomizing the order of operand loading and storing within each butterfly operation. An arithmetic conditional swap is used to decide the order based on a random bit. This method is effective against attacks that target specific operand accesses.

Bitwise Fine Shuffling enhances this by breaking down operations into single-bit manipulations, further obscuring the side-channel leakage. However, the increased computational complexity can significantly impact performance, especially on resource-constrained devices.

\subsubsection{Effectiveness Against NTT-Based Attacks}

Shuffling the \ac{NTT} operations disrupts the deterministic execution patterns exploited by attacks such as those by Primas et al.~\cite{Primas17} and Pessl et al.~\cite{Pessl19}. By randomizing computation order, shuffling makes it harder for attackers to map side-channel emissions to specific \ac{NTT} operations, reducing key recovery risks.

Ravi et al.~\cite{Ravi20} evaluated their shuffling countermeasures on the ARM Cortex-M4 microcontroller, demonstrating that these techniques can be practically implemented on embedded devices. They also analyzed the performance overhead introduced by each shuffling method, providing insights into the trade-offs between security and efficiency.