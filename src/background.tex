\chapter{Background}
\thispagestyle{chapterstart}

\section{CRYSTALS-Dilithium Digital Signature Scheme}

CRYSTALS-Dilithium is a lattice-based digital signature scheme that has emerged as a leading candidate in the \ac{NIST} Post-Quantum Cryptography standardization process. It is constructed on the hardness assumptions of the \acp{MLWE} and \acp{MSIS} problems, which are believed to be resistant to attacks even by quantum computers.

Dilithium operates over structured lattices and employs efficient polynomial arithmetic in the ring $\mathbb{Z}_q[X]/(X^n + 1)$, where $q$ is a prime modulus and $n$ is a power of two. Key operations in Dilithium include polynomial multiplication and addition modulo $(X^n + 1)$ and $q$, sampling of random polynomials with coefficients from specific distributions, and decomposition functions used in signature generation and verification.

To achieve high performance, Dilithium implementations often utilize the \acp{NTT}, a specialized version of the \acp{FFT} adapted for finite fields. The NTT allows efficient polynomial multiplication by transforming polynomials into the frequency domain, where convolution becomes point-wise multiplication, significantly reducing computational complexity.

\section{Side-Channel Attacks on Cryptographic Implementations}

Side-channel attacks exploit unintentional information leakage from physical implementations of cryptographic algorithms to recover sensitive information such as secret keys. These leakages can stem from variations in power consumption, electromagnetic emissions, timing information, or other physical phenomena correlated with the processed data or executed operations.

Embedded devices are particularly susceptible to side-channel attacks due to their physical accessibility and resource constraints, which may limit the implementation of countermeasures. Ensuring that cryptographic implementations, especially those intended for deployment on embedded platforms, are resistant to side-channel attacks is essential for maintaining security in practical applications.

\section{Masking as a Side-Channel Countermeasure}

Masking is a widely used countermeasure against side-channel attacks that involves representing sensitive variables as a combination of random shares. The fundamental idea is to split a secret value into multiple shares such that each share individually provides no information about the secret. For example, in Boolean masking, a secret value $x$ is split into $t+1$ shares $(x_0, x_1, \dots, x_t)$ satisfying $x = x_0 \oplus x_1 \oplus \dots \oplus x_t$.

By processing each share separately and ensuring that any side-channel leakage is independent of the actual secret value, masking can significantly reduce the effectiveness of side-channel attacks. Masking schemes are analyzed under various security models, such as the $t$-probing model, which formalizes the adversary's capabilities and the required level of protection.

Implementing masking in practice, especially on embedded devices, requires careful consideration of performance overheads and the complexity of securely handling operations involving masked values, particularly non-linear functions.

\section{Shuffling as a Side-Channel Countermeasure}

Shuffling is another countermeasure against side-channel attacks, which aims to randomize the order of operations or the processing of data elements to prevent attackers from correlating side-channel observations with specific computations. By introducing randomness into the execution flow, shuffling disrupts the deterministic patterns that side-channel attacks often exploit.

In the context of cryptographic algorithms like Dilithium, shuffling can be applied to operations such as the processing of polynomial coefficients or the execution order of arithmetic computations. For instance, the order in which the butterfly operations in the Number Theoretic Transform (NTT) are performed can be randomized.

Shuffling techniques include:

\begin{itemize}
    \item \textbf{Coarse Shuffling}: Randomizing the order of larger blocks of operations, which provides moderate security improvements with relatively low performance overhead.
    \item \textbf{Fine Shuffling}: Randomizing the order of individual operations or data elements, offering higher security at the cost of increased computational complexity.
\end{itemize}

Implementing shuffling effectively requires generating high-quality randomness and carefully managing the additional computational overhead to maintain acceptable performance on embedded devices. While shuffling alone may not provide as strong protection as higher-order masking, it can be an effective component of a composite countermeasure strategy, particularly when combined with other techniques.

