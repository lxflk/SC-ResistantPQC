\chapter{Introduction}
\thispagestyle{chapterstart}

\section{Motivation}

The emergence of large-scale quantum computers threatens to undermine the security of currently deployed asymmetric cryptographic schemes, such as RSA and \ac{ECC}, which rely on mathematical problems that quantum algorithms can solve efficiently. To safeguard digital communications and data, quantum-resistant cryptographic schemes are being developed and standardized since 2016. One of the leading candidates in this area is CRYSTALS-Dilithium, a lattice-based digital signature algorithm offering strong theoretical security against quantum attacks.

However, transitioning these post-quantum cryptographic (PQC) schemes from theory to practice introduces significant challenges, particularly regarding their implementation on embedded devices. Resource constraints and the physical exposure of embedded systems make them especially vulnerable to side-channel attacks, which exploit physical leakages such as power consumption and electromagnetic emissions to extract secret information. Ensuring that PQC implementations are resistant to such attacks is crucial for their secure deployment in real-world applications.

\section{Objectives and Approach}

The primary objective of this paper is to explore the feasibility and performance of side-channel resistant implementations of post-quantum cryptographic schemes on embedded devices. Specifically, we aim to:

\begin{itemize}
    \item \textbf{Identify Key Aspects}: Introduce and analyze the most important aspects that affect the feasibility and performance of side-channel resistant implementations of \ac{PQC} schemes, focusing on vulnerabilities inherent in algorithms like Dilithium.
    \item \textbf{Examine Countermeasures}: Investigate various countermeasures against side-channel attacks, including masking and shuffling techniques, and evaluate their effectiveness and impact on performance.
    \item \textbf{Compare Implementations}: Visualize, compare, and discuss existing side-channel resistant implementations of \ac{PQC} schemes, particularly Dilithium, with respect to the identified aspects.
    \item \textbf{Assess Trade-offs}: Analyze the trade-offs between security and performance in these implementations, providing insights into their practicality for deployment on embedded devices.
\end{itemize}

Through this comprehensive analysis, we aim to provide guidance for practitioners and researchers in selecting and optimizing side-channel countermeasures for PQC implementations in resource-constrained environments.

\section{Structure of the Paper}

The paper is organized as follows:

\begin{itemize}
    \item \textbf{Chapter 2 - Background}: Provides an overview of post-quantum cryptography and the CRYSTALS-Dilithium scheme, and introduces side-channel attacks and common countermeasures.
    \item \textbf{Chapter 3 - Analysis}: Analyzes side-channel vulnerabilities in Dilithium, focusing on specific weaknesses such as the bit-unpacking function and the Number Theoretic Transform (NTT), and examines initial and improved countermeasures.
    \item \textbf{Chapter 4 - Comparative Analysis}: Visualizes and compares various side-channel resistant implementations of Dilithium, discussing their feasibility and performance on embedded devices.
    \item \textbf{Chapter 5 - Conclusion}: Summarizes the findings, discusses the implications for deploying side-channel resistant PQC schemes, and suggests directions for future research.
\end{itemize}

This structure enables a systematic exploration from theoretical considerations to practical implementation challenges, culminating in actionable recommendations for enhancing the security and efficiency of PQC schemes on embedded platforms.
