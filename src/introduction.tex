% introduction.tex

\chapter{Introduction}
\thispagestyle{chapterstart}

\section{Motivation}

The rapid advancement of quantum computing technology poses a significant threat to current cryptographic systems, particularly those based on integer factorization and discrete logarithm problems, such as RSA and \ac{ECC}. Quantum algorithms like Shor's algorithm can solve these problems efficiently, rendering classical cryptosystems insecure. To address this threat, the \ac{NIST} initiated a standardization process for \ac{PQC} algorithms that are secure against both quantum and classical attacks.

Among the winning candidates in this process is the lattice-based scheme CRYSTALS-Dilithium, known for its strong security foundations and efficient performance. However, while Dilithium is resistant to quantum attacks, its implementations on embedded devices are vulnerable to \acp{SCA}, which exploit physical leakages to extract sensitive information.

Side-channel resistance is critical for embedded devices, often with limited resources and physical accessibility. Enhancing the side-channel resistance of Dilithium implementations is essential for securing future cryptographic systems in the quantum era.

\section{Goals}

The primary goal of this paper is to analyze and compare various side-channel resistant implementations of the Dilithium digital signature scheme, focusing on methodologies, security levels, performance, and feasibility on embedded devices. Specifically, the objectives are:

\begin{itemize}
    \item \textbf{In-depth Analysis}: Examine selected implementations, highlighting techniques used for security and performance improvements.
    \item \textbf{Comparative Evaluation}: Compare implementations in terms of security levels (masking orders), performance metrics, and practical considerations.
    \item \textbf{Discussion of Trade-offs}: Identify and discuss trade-offs between security and performance in different masking techniques.
    \item \textbf{Insights for Deployment}: Offer strategies for deploying secure and efficient Dilithium implementations in resource-constrained environments.
\end{itemize}

By achieving these goals, this paper aims to contribute to understanding side-channel resistant implementations of Dilithium and guide future research.

\section{Structure of the Paper}

This paper is structured as follows:

\begin{itemize}
    \item \textbf{Chapter 2 - Background}: Provides an overview of post-quantum cryptography and quantum threats, introduces side-channel attacks, and discusses masking techniques relevant to Dilithium.
    \item \textbf{Chapter 3 - Analysis}: Presents an in-depth analysis of selected side-channel resistant implementations of Dilithium, examining methodologies and practical considerations.
    \item \textbf{Chapter 4 - Comparative Analysis}: Compares the analyzed implementations, highlighting strengths and weaknesses, discussing trade-offs, and implications for deployment.
    \item \textbf{Chapter 5 - Conclusion}: Summarizes the findings, discusses implications for deploying side-channel resistant Dilithium implementations, and suggests directions for future research.
\end{itemize}

By following this structure, the paper builds from foundational concepts to detailed analysis and comparison, culminating in conclusions that inform both academic and practical perspectives.
