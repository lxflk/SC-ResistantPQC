\chapter{Introduction}
\thispagestyle{chapterstart}

\section{Motivation}

The advent of quantum computing presents a formidable challenge to existing cryptographic systems, especially to widely used public-key algorithms such as RSA and \ac{ECC}. These algorithms are rendered vulnerable by quantum algorithms like Shor's, which can solve integer factorization and discrete logarithm problems in polynomial time. As a result, the \ac{NIST} has initiated a standardization process to identify \ac{PQC} algorithms that are secure against both quantum and classical adversaries \cite{NIST24}.

While lattice-based schemes such as Kyber and Dilithium have emerged as leading candidates due to their strong security assumptions and efficiency, they are also susceptible to \acp{SCA}, particularly on embedded devices. These devices are often constrained in resources and may be physically accessible to attackers, making them more vulnerable to \acp{SCA}, which exploit physical characteristics of cryptographic implementations—such as power consumption and electromagnetic emissions—to uncover sensitive information \cite{Bos21}. This is a critical concern as \acp{SCA} can compromise security even if the underlying cryptographic algorithms remain theoretically sound against quantum attacks.

The importance of side-channel resistance in \ac{PQC} implementations has led to extensive research on countermeasures. For instance, Migliore et al.\ \cite{Migliore19} introduced masking techniques for Dilithium to reduce side-channel leakage, while Heinz et al.\ \cite{Heinz20} presented a first-order masked implementation of Kyber on the \ac{ARM} Cortex-M4 processor, demonstrating the practicality of side-channel resistance on embedded platforms. More recent works, such as those by Bronchain and Cassiers \cite{Bronchain22} and Coron et al.\ \cite{Coron23}, have further advanced the field by optimizing masking conversions and introducing high-order masking gadgets that significantly improve security and efficiency. This seminar paper seeks to build on these contributions by analyzing the performance, security, and practicality of various side-channel resistant implementations for \ac{PQC} schemes.

\section{Goals}

The primary goal of this seminar paper is to systematically examine and evaluate existing side-channel resistant implementations of lattice-based \ac{PQC} schemes on embedded systems. The key objectives are as follows:

\begin{itemize}
    \item \textbf{Introduction of Key Concepts:} Provide an in-depth overview of critical aspects that affect the feasibility, performance, and security of side-channel resistant \ac{PQC} implementations, including masking techniques, bitslicing, and hardware considerations.
    \item \textbf{Comparative Analysis:} Systematically compare implementations of \ac{PQC} schemes like Kyber and Dilithium concerning their side-channel resistance, performance metrics, and resource utilization on embedded devices.
    \item \textbf{Evaluation of Countermeasures:} Assess the effectiveness of various side-channel countermeasures, including first-order and higher-order masking, bitslicing, and optimized gadgets, in protecting against \acp{SCA}.
    \item \textbf{Identification of Trade-offs:} Identify and discuss the trade-offs between security and efficiency in side-channel resistant implementations, offering insights into optimal strategies for secure and performant \ac{PQC} deployment.
\end{itemize}

By achieving these goals, this paper aims to contribute to a comprehensive understanding of the current landscape of side-channel resistant \ac{PQC} implementations, highlighting the advancements and challenges in deploying quantum-resistant cryptographic solutions in resource-constrained environments.

\section{Structure of the Paper}

This paper is structured as follows:
\begin{itemize}
    \item \textbf{Chapter 2 - Background:} A comprehensive overview of post-quantum cryptography, the quantum threat to classical cryptosystems, and an introduction to side-channel attacks. This chapter also covers the primary countermeasures developed for enhancing the side-channel resistance of \ac{PQC} implementations.
    \item \textbf{Chapter 3 - Analysis:} An analysis of the methodology used to evaluate side-channel resistance and a comparative analysis of various implementations of \ac{PQC} schemes, such as Kyber and Dilithium, in terms of security level, performance, and practicality.
    \item \textbf{Chapter 4 - Conclusion:} A summary of the findings, a discussion of the trade-offs between security and efficiency, and an outlook on potential future directions in side-channel resistance for \ac{PQC} schemes.
\end{itemize}
