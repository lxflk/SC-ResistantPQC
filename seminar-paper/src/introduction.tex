% introduction.tex

\chapter{Introduction}
\thispagestyle{chapterstart}

\section{Motivation}

The rapid advancement of quantum computing technology poses a significant threat to current cryptographic systems, particularly those based on integer factorization and discrete logarithm problems, such as RSA and \ac{ECC}. Quantum algorithms like Shor's algorithm can solve these problems efficiently, rendering classical cryptosystems insecure in the presence of a sufficiently powerful quantum computer. To address this looming threat, the \ac{NIST} initiated a standardization process for \ac{PQC} algorithms that are secure against both quantum and classical attacks.

Among the leading candidates in this standardization effort are lattice-based schemes like CRYSTALS-Dilithium, a digital signature algorithm known for its strong security foundations and efficient performance on classical computing architectures. However, while Dilithium is designed to be resistant to attacks from quantum computers, its implementations on embedded devices are vulnerable to \acp{SCA}. These attacks exploit physical leakages, such as power consumption or electromagnetic emissions, to extract sensitive information from cryptographic operations.

Side-channel resistance is especially critical for embedded devices, which often have limited computational resources and are deployed in environments where attackers may have physical access. As a result, enhancing the side-channel resistance of Dilithium implementations is essential for ensuring the security of future cryptographic systems in the quantum era.

\section{Goals}

The primary goal of this seminar paper is to analyze and compare various side-channel resistant implementations of the Dilithium digital signature scheme, focusing on their methodologies, security levels, performance, and feasibility on embedded devices. Specifically, the objectives are:

\begin{itemize}
    \item \textbf{In-depth Analysis}: Provide a detailed examination of selected implementations of side-channel resistant Dilithium, highlighting the techniques used to achieve security and performance improvements.
    \item \textbf{Comparative Evaluation}: Systematically compare the implementations in terms of security levels (masking orders), performance metrics (execution time, operation counts), and practical considerations for embedded systems.
    \item \textbf{Discussion of Trade-offs}: Identify and discuss the trade-offs between security and performance inherent in different masking techniques and implementations.
    \item \textbf{Insights for Deployment}: Offer insights into optimal strategies for deploying secure and efficient Dilithium implementations in real-world, resource-constrained environments.
\end{itemize}

By achieving these goals, this paper aims to contribute to a better understanding of the current state of side-channel resistant implementations of Dilithium and provide guidance for future research and development in this area.

\section{Structure of the Paper}

This paper is structured as follows:

\begin{itemize}
    \item \textbf{Chapter 2 - Background}: Provides an overview of post-quantum cryptography and the quantum threats to classical cryptosystems. It introduces side-channel attacks and discusses common countermeasures, with a focus on masking techniques relevant to Dilithium.
    \item \textbf{Chapter 3 - Analysis}: Presents an in-depth analysis of selected side-channel resistant implementations of Dilithium. It examines the methodologies, security proofs, performance metrics, and practical considerations of each implementation.
    \item \textbf{Chapter 4 - Comparative Analysis}: Compares the analyzed implementations, highlighting their strengths and weaknesses. It discusses the trade-offs between security levels and performance, and their implications for practical deployment.
    \item \textbf{Chapter 5 - Conclusion}: Summarizes the findings, discusses the implications for deploying side-channel resistant Dilithium implementations, and suggests directions for future research.
\end{itemize}

By following this structure, the paper systematically builds from foundational concepts to detailed analysis and comparison, culminating in conclusions that inform both academic and practical perspectives on the topic.

