% conclusion.tex

\chapter{Conclusion}
\thispagestyle{chapterstart}

\section{Summary of Findings}

This paper has analyzed and compared three significant implementations of side-channel resistant Dilithium:

\begin{enumerate}
    \item \textbf{Prest et al.\ (2021)}: Focused on efficient first-order masking techniques for Dilithium, demonstrating acceptable performance on embedded devices.
    \item \textbf{Azouaoui et al.\ (2022)}: Extended masking to higher orders, up to \( d = 8 \), and introduced optimized gadgets. They highlighted the benefits of randomized signing over deterministic signing in terms of both security and performance.
    \item \textbf{Coron et al.\ (2023)}: Introduced novel gadgets like ShiftMod for efficient high-order masking, providing strong security proofs and practical performance improvements for small masking orders.
\end{enumerate}

Our analysis revealed that while all three implementations aim to enhance side-channel resistance, they differ in their approaches, targeted security levels, and performance trade-offs. Prest et al.\ prioritize performance for first-order masking, making it suitable for applications where minimal overhead is acceptable. Azouaoui et al.\ provide scalable higher-order masking, emphasizing the importance of refined sensitivity analysis and the advantages of randomized signing. Coron et al.\ offer innovative solutions for high-order masking with strong security guarantees but acknowledge limitations in practicality for large masking orders due to exponential complexity.

\section{Discussion of Trade-offs}

The primary trade-offs identified between security and performance include:

\begin{itemize}
    \item \textbf{Masking Order vs. Performance}: Higher-order masking provides stronger security against sophisticated side-channel attacks but incurs significant computational overhead, making it less practical for resource-constrained devices.
    \item \textbf{Randomized vs. Deterministic Signing}: Randomized signing reduces the need for masking complex functions like hash operations, leading to better performance and security. However, deterministic signing may be preferred in applications requiring reproducibility.
    \item \textbf{Algorithm Complexity}: Innovative gadgets like those introduced by Coron et al.\ improve efficiency for small masking orders but may not scale well for higher orders due to increased complexity.
\end{itemize}

These trade-offs highlight the need to balance security requirements with practical constraints, choosing appropriate masking techniques based on the specific application and acceptable performance overhead.

\section{Implications for Deployment}

For practitioners and developers aiming to implement side-channel resistant Dilithium in real-world applications, the following considerations are important:

\begin{itemize}
    \item \textbf{Application Requirements}: Assess the required security level and acceptable performance overhead. For applications where first-order side-channel resistance suffices, methods like those of Prest et al.\ may be adequate.
    \item \textbf{Resource Constraints}: Evaluate the computational and memory resources of the target platform. High-order masking may not be feasible on devices with limited capabilities.
    \item \textbf{Security Proofs}: Consider implementations with strong security proofs, especially for applications requiring high assurance, even if it comes at the cost of increased complexity.
    \item \textbf{Randomized Signing}: Where acceptable, prefer randomized signing to leverage performance and security benefits, as demonstrated by Azouaoui et al.
\end{itemize}

\section{Future Work}

Future research can explore:

\begin{itemize}
    \item \textbf{Scalable High-Order Masking}: Develop methods to reduce the complexity of high-order masking gadgets, making them practical for higher masking orders.
    \item \textbf{Hardware Support}: Investigate hardware acceleration and support for masking operations to improve performance on embedded devices.
    \item \textbf{Unified Frameworks}: Create unified frameworks that combine the strengths of various approaches, offering configurable security levels and optimized performance.
    \item \textbf{Side-Channel Analysis Tools}: Enhance tools for side-channel analysis and verification to aid in the development of secure implementations.
\end{itemize}