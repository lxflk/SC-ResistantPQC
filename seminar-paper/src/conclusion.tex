% conclusion.tex

\chapter{Conclusion}
\thispagestyle{chapterstart}

\section{Summary of Findings}

This paper has analyzed and compared three significant implementations of side-channel resistant Dilithium:

\begin{enumerate}
    \item \textbf{Migliore et al.\ (2019) \cite{Migliore19}}: Focused on efficient first-order masking techniques, proposing a power-of-two modulus to improve efficiency. However, their sensitivity analysis had flaws leading to potential insecurities and inefficiencies.
    \item \textbf{Azouaoui et al.\ (2022) \cite{Azouaoui22}}: Revisited the sensitivity analysis, correcting misconceptions, and introduced improved masking gadgets. They extended masking to higher orders up to \( d = 8 \) and highlighted the benefits of randomized signing.
    \item \textbf{Coron et al.\ (2023) \cite{Coron23}}: Introduced novel gadgets like ShiftMod for efficient high-order masking, providing strong security proofs and practical improvements for small masking orders. They acknowledged scalability limitations at higher orders.
\end{enumerate}

Our analysis revealed that while all aim to enhance side-channel resistance, they differ in approaches, targeted security levels, and performance trade-offs. Migliore et al.\ prioritize performance for first-order masking but modify the scheme, affecting standardization. Azouaoui et al.\ provide scalable higher-order masking, emphasizing refined sensitivity analysis to avoid insecurity and overhead. Coron et al.\ offer innovative solutions with strong security guarantees but face practical limitations due to increased complexity at higher masking orders.

\section{Discussion of Trade-offs}

The primary trade-offs between security and performance include:

\begin{itemize}
    \item \textbf{Masking Order vs.\ Performance}: Higher-order masking provides stronger security but incurs significant overhead.
    \item \textbf{Modification of the Scheme}: Migliore et al.'s use of a power-of-two modulus simplifies masking but may affect compliance and security proofs.
    \item \textbf{Randomized vs.\ Deterministic Signing}: Randomized signing reduces the need for masking complex computations, leading to better performance and security.
    \item \textbf{Algorithm Complexity}: Innovative gadgets improve efficiency at small masking orders but may not scale well due to complexity.
\end{itemize}

These trade-offs highlight the need to balance security requirements with practical constraints.

\section{Implications for Deployment}

For implementing side-channel resistant Dilithium:

\begin{itemize}
    \item \textbf{Assess Security Needs}: Determine the required masking order based on the threat model.
    \item \textbf{Consider Standard Compliance}: Be cautious about modifying the original scheme.
    \item \textbf{Optimize for Resources}: Use optimized gadgets and randomized signing to make higher-order masking feasible.
    \item \textbf{Prioritize Security Proofs}: Implementations with strong security proofs provide higher assurance.
    \item \textbf{Prefer Randomized Signing}: When acceptable, to leverage performance and security benefits.
\end{itemize}

\section{Future Work}

Future research can explore:

\begin{itemize}
    \item \textbf{Optimizing Masking Gadgets}: Develop more efficient gadgets scalable to higher orders.
    \item \textbf{Alternative Schemes}: Investigate other post-quantum signature schemes for better trade-offs.
    \item \textbf{Hardware Support}: Explore hardware acceleration for masking operations.
    \item \textbf{Combined Countermeasures}: Integrate masking with other techniques to enhance security.
    \item \textbf{Comprehensive Security Analysis}: Include both side-channel and fault injection attacks in analyses.
\end{itemize}
