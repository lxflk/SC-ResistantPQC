% EMSEC seminar template, 24.09.2010
% david.oswald@rub.de
%
% based on
% EMSEC thesis template, 13.09.2010
%
% Calls all chapters
% Each chapter is contained in a separate "\input" file
% Style: doublesided, scrreport, dina4
%
% Benedikt Driessen, 2010
% benedikt.driessen@rub.de
%

% *************************************************************
% ENTER INFORMATION ABOUT AUTHOR, TITLE and TYPE HERE
\newcommand{\thauthor}{Forename Surname}
\newcommand{\thtitle}{Title}
% *************************************************************

\documentclass[a4paper,12pt,twoside,openany,headsepline,bibliography=totocnumbered]{scrbook}

% Import settings, packages, etc.
% settings.tex

\newcommand{\thauthor}{Luis Falke}
\newcommand{\thtitle}{Side-Channel Resistant Implementations of Post-Quantum Cryptography}

\usepackage[T1]{fontenc}
\usepackage[utf8]{inputenc}
\usepackage{graphicx}
\usepackage{epstopdf}
\usepackage{amsmath}
\usepackage{amssymb}
\usepackage{amsfonts}
\usepackage{eucal}
\usepackage{fancyhdr}
\usepackage{url}
\usepackage{listings}
\usepackage[printonlyused]{acronym}
\usepackage{algorithmic}
\usepackage{algorithm}
\usepackage{multicol}
\usepackage{hyphenat}
\usepackage{cite}
\usepackage{subcaption}
\usepackage{etoolbox}

\makeatletter
\patchcmd{\chapter}{plain}{fancyplain}{}{}
\makeatother

% Custom styles for different sections
\fancypagestyle{chapterstart}{
    \fancyhf{}
    \fancyhead[L]{\slshape \leftmark}
    \fancyhead[R]{\thepage}
    \renewcommand{\headrulewidth}{0.4pt}
}

\fancypagestyle{abstract}{
    \fancyhf{}
    \fancyhead[L]{\slshape Abstract}
    \fancyhead[R]{\thepage}
    \renewcommand{\headrulewidth}{0.4pt}
}

\fancypagestyle{contents}{
    \fancyhf{}
    \fancyhead[L]{\slshape Table of Contents}
    \fancyhead[R]{\thepage}
    \renewcommand{\headrulewidth}{0.4pt}
}

\fancypagestyle{acronyms}{
    \fancyhf{}
    \fancyhead[L]{\slshape Acronyms}
    \fancyhead[R]{\thepage}
    \renewcommand{\headrulewidth}{0.4pt}
}

\fancypagestyle{listsoffigures}{
    \fancyhf{}
    \fancyhead[L]{\slshape List of Figures}
    \fancyhead[R]{\thepage}
    \renewcommand{\headrulewidth}{0.4pt}
}

\fancypagestyle{listoftables}{
    \fancyhf{}
    \fancyhead[L]{\slshape List of Tables}
    \fancyhead[R]{\thepage}
    \renewcommand{\headrulewidth}{0.4pt}
}

\fancypagestyle{bibliography}{
    \fancyhf{}
    \fancyhead[L]{\slshape Bibliography}
    \fancyhead[R]{\thepage}
    \renewcommand{\headrulewidth}{0.4pt}
}

% Graphics path
\graphicspath{{../figures/}}

% URL style
\urlstyle{tt}

% Separation between list items
\setlength{\itemsep}{0ex plus0.2ex}

% Fancy headers
\setlength{\headsep}{8mm}
\pagestyle{fancyplain}

% Only chapter title in header
\renewcommand{\chaptermark}[1]{\markboth{\thechapter.\ #1}{}}
\renewcommand{\sectionmark}[1]{}

% Left header shows chapter title, right header shows page number
\lhead[\fancyplain{}{\thepage}]{\fancyplain{}{\slshape \leftmark}}
\rhead[\fancyplain{}{\slshape \leftmark}]{\fancyplain{}{\thepage}}
\cfoot{}


\begin{document}

% Import frontpage
% frontpage.tex

\begin{titlepage}
    \enlargethispage{3cm}
    \vspace*{-32mm}\hspace*{120mm}
    \includegraphics[scale=1.0]{rub_logo-eps-converted-to.pdf}
    
    \vspace*{11cm}\hspace*{0mm}
    \begin{minipage}[b]{1\linewidth}
        \sffamily
        \hspace{-17.2mm}\includegraphics[scale=1.0]{rub_slogan-eps-converted-to.pdf}\\
        
        \nohyphens{
            {\bfseries \LARGE \sffamily {\thtitle}}
        }\\
        
        \large{
            \thauthor
        }\\
        
        \vspace*{35mm}
        \normalsize{
            Seminar Paper\\
            \today\\
            Chair for Security Engineering - Prof. Dr.-Ing. Tim G{\"u}neysu\\
            Advisor: Georg Land
        }
    \end{minipage}
\end{titlepage}


% Abstract
\section*{Abstract}
The following text\footnote{Shamelessly ripped from \url{http://www.leeds.ac.uk/educol/abstract.htm}} describes what your abstract should be about.

The following text\footnote{Shamelessly ripped from \url{http://www.leeds.ac.uk/educol/abstract.htm}} describes what your abstract should be about.

The abstract should convey to the reader concisely and accurately within the space of a few sentences, the claim to knowledge that the authors are making. It should indicate the boundaries of space and time within which the enquiry has occurred. If there is a claim to generality beyond the boundaries of the enquiry the basis of that claim should be given, for example that a random sample is thought to be representative of a larger population. There should also be a hint of the method of enquiry.

The boundaries of an enquiry are important - and are unfortunately too often omitted from abstracts. This is due to the regrettable tendency for researchers to generalise their results from, for example, a few schools to all schools, and to imply that what was true at a particular time, is true for all time. Some reference to the geographical location of the children, or teachers, or schools on whom the claim to knowledge rests should be made. Because of the international nature of the research community it is worth making clear in what country the research took place. Also the period in which the data was collected should be stated.

The abstract should be a condensation of the substance of the paper, not a trailer, nor an introduction. Journals and thesis regulations usually put a limit of around 200 to 300 words to the length of an abstract. “Trailer” is a term borrowed from the cinema industry to describe a showing of a few highlights in order to win an audience. An “Introduction” tells that something is coming, but doesn’t reveal its substance. These are not what is needed.

Abstracts are recycled in abstract journals and electronic networks and provide the main vehicle for other researchers to become aware of particular studies. Hence the more clearly they convey the claim to knowledge of the original paper the more useful they are in helping the reader to decide whether it is worth taking the trouble to obtain and read the original and possibly cite it in his/her own writing.

Both the abstract and the paper should make sense without the other.

\clearpage

\tableofcontents
\mainmatter

% List of acronyms
\chapter*{Acronyms}
\begin{acronym}
    \setlength{\itemsep}{0.2em}
    \acro{ACR}{ACRONYM1}
    % etc.
\end{acronym}

% Include all your chapters as .tex files
\pagenumbering{arabic}

\chapter{Introduction}
\thispagestyle{chapterstart}

\section{Motivation}

The emergence of large-scale quantum computers threatens to undermine the security of currently deployed asymmetric cryptographic schemes, such as RSA and \ac{ECC}, which rely on mathematical problems that quantum algorithms can solve efficiently. To safeguard digital communications and data, quantum-resistant cryptographic schemes are being developed and standardized since 2016. One of the leading candidates in this area is CRYSTALS-Dilithium, a lattice-based digital signature algorithm offering strong theoretical security against quantum attacks.

However, transitioning these post-quantum cryptographic (PQC) schemes from theory to practice introduces significant challenges, particularly regarding their implementation on embedded devices. Resource constraints and the physical exposure of embedded systems make them especially vulnerable to side-channel attacks, which exploit physical leakages such as power consumption and electromagnetic emissions to extract secret information. Ensuring that PQC implementations are resistant to such attacks is crucial for their secure deployment in real-world applications.

\section{Objectives and Approach}

The primary objective of this paper is to explore the feasibility and performance of side-channel resistant implementations of post-quantum cryptographic schemes on embedded devices. Specifically, we aim to:

\begin{itemize}
    \item \textbf{Identify Key Aspects}: Introduce and analyze the most important aspects that affect the feasibility and performance of side-channel resistant implementations of \ac{PQC} schemes, focusing on vulnerabilities inherent in algorithms like Dilithium.
    \item \textbf{Examine Countermeasures}: Investigate various countermeasures against side-channel attacks, including masking and shuffling techniques, and evaluate their effectiveness and impact on performance.
    \item \textbf{Compare Implementations}: Visualize, compare, and discuss existing side-channel resistant implementations of \ac{PQC} schemes, particularly Dilithium, with respect to the identified aspects.
    \item \textbf{Assess Trade-offs}: Analyze the trade-offs between security and performance in these implementations, providing insights into their practicality for deployment on embedded devices.
\end{itemize}

Through this comprehensive analysis, we aim to provide guidance for practitioners and researchers in selecting and optimizing side-channel countermeasures for PQC implementations in resource-constrained environments.

\section{Structure of the Paper}

The paper is organized as follows:

\begin{itemize}
    \item \textbf{Chapter 2 - Background}: Provides an overview of post-quantum cryptography and the CRYSTALS-Dilithium scheme, and introduces side-channel attacks and common countermeasures.
    \item \textbf{Chapter 3 - Analysis}: Analyzes side-channel vulnerabilities in Dilithium, focusing on specific weaknesses such as the bit-unpacking function and the Number Theoretic Transform (NTT), and examines initial and improved countermeasures.
    \item \textbf{Chapter 4 - Comparative Analysis}: Visualizes and compares various side-channel resistant implementations of Dilithium, discussing their feasibility and performance on embedded devices.
    \item \textbf{Chapter 5 - Conclusion}: Summarizes the findings, discusses the implications for deploying side-channel resistant PQC schemes, and suggests directions for future research.
\end{itemize}

This structure enables a systematic exploration from theoretical considerations to practical implementation challenges, culminating in actionable recommendations for enhancing the security and efficiency of PQC schemes on embedded platforms.

%\input{theory}
%\input{design}
%\input{implementation}
%\input{results}
\chapter{Conclusion}
\thispagestyle{chapterstart}

The transition to post-quantum cryptography is imperative in the face of advancing quantum computing capabilities that threaten the security of classical cryptographic schemes. However, implementing post-quantum algorithms like CRYSTALS-Dilithium securely on embedded devices presents significant challenges due to side-channel vulnerabilities inherent in practical implementations.

In this paper, we analyzed the critical aspects affecting the feasibility and performance of side-channel resistant implementations of post-quantum schemes, focusing on Dilithium. We identified key vulnerabilities, such as those in the bit-unpacking function and the Number Theoretic Transform (NTT), which can be exploited to recover secret keys through side-channel attacks. To address these vulnerabilities, we examined various countermeasures, including initial masking techniques, subsequent improvements with optimized masking gadgets, and shuffling methods.

Through a comparative analysis, we visualized and discussed existing implementations regarding their security and performance on embedded devices. Our findings indicate that while higher-order masking techniques offer strong protection against side-channel attacks, they often introduce substantial performance overheads, making them less feasible for resource-constrained environments. Conversely, shuffling techniques and optimized masking methods can provide a more favorable balance between security and efficiency.

The trade-offs between security and performance are critical when selecting appropriate countermeasures for PQC implementations on embedded devices. Practitioners must consider the specific security requirements and constraints of their applications. Our study highlights the importance of continued research and development in optimizing side-channel countermeasures, exploring hybrid approaches that combine multiple techniques, and tailoring implementations to the capabilities of target devices.

In conclusion, enhancing the side-channel resistance of post-quantum cryptographic schemes is essential for their secure deployment in the quantum era. By carefully selecting and optimizing countermeasures, it is possible to achieve practical and secure PQC implementations on embedded systems. Future work should focus on further reducing the performance impact of countermeasures, developing standardized methodologies for side-channel resistance, and ensuring that security evaluations keep pace with evolving attack strategies.


% Generate list of figures
\newpage
\listoffigures

% Generate list of tables
\newpage
\listoftables
\clearpage

% Start back matter
\backmatter

% Generate bibliography with BibTeX
\bibliographystyle{alpha}
\bibliography{bibliography}

\end{document}
