% EMSEC seminar template, 24.09.2010
% david.oswald@rub.de
%
% based on
% EMSEC thesis template, 13.09.2010
%
% Calls all chapters
% Each chapter is contained in a separate "\input" file
% Style: doublesided, scrreport, dina4
%
% Benedikt Driessen, 2010
% benedikt.driessen@rub.de
%

% *************************************************************
% ENTER INFORMATION ABOUT AUTHOR, TITLE and TYPE HERE
\newcommand{\thauthor}{Forename Surname}
\newcommand{\thtitle}{Title}
% *************************************************************

\documentclass[a4paper,12pt,twoside,openany,headsepline,bibliography=totocnumbered]{scrbook}

% Import settings, packages, etc.
% List of packages and settings

\usepackage[T1]{fontenc}
\usepackage[utf8]{inputenc}
\usepackage{graphicx}
\usepackage{epstopdf}
\usepackage{amsmath}
\usepackage{amssymb}
\usepackage{amsfonts}
\usepackage{eucal}
\usepackage{fancyhdr}
\usepackage{url}
\usepackage{listings}
\usepackage[printonlyused]{acronym}
\usepackage{algorithmic}
\usepackage{algorithm}
\usepackage{multicol}
\usepackage{hyphenat}

% Set graphics path
\graphicspath{{../figures/}}

% URL style
\urlstyle{tt}

% Separation between list items
\setlength{\itemsep}{0ex plus0.2ex}

% Fancy headers
\setlength{\headsep}{8mm}
\pagestyle{fancyplain}
\renewcommand{\chaptermark}[1]{\markboth{#1}{#1}}
\renewcommand{\sectionmark}[1]{\markright{\thesection\ #1}}
\lhead[\fancyplain{}{\thepage}]{\fancyplain{}{\slshape \rightmark}}
\rhead[\fancyplain{}{\slshape \leftmark}]{\fancyplain{}{\thepage}}
\cfoot{}


\begin{document}

% Import frontpage
% Frontpage

\frontmatter

\begin{titlepage}
    \enlargethispage{3cm}
    \vspace*{-32mm}\hspace*{120mm}
    \includegraphics[scale=1.0]{rub_logo-eps-converted-to.pdf}
    
    \vspace*{11cm}\hspace*{0mm}
    \begin{minipage}[b]{1\linewidth}
        \sffamily
        \hspace{-17.2mm}\includegraphics[scale=1.0]{rub_slogan-eps-converted-to.pdf}\\
        
        \nohyphens{
            {\bfseries \LARGE \sffamily {\thtitle}}
        }\\
        
        \large{
            \thauthor
        }\\
        
        \vspace*{35mm}
        \normalsize{
            Seminar Paper\\
            \today\\
            Chair for Security Engineering - Prof. Dr.-Ing. Tim G{\"u}neysu\\
            Advisor: Georg Land
        }
    \end{minipage}
\end{titlepage}

\newpage\thispagestyle{empty}


% Abstract
\section*{Abstract}
The following text\footnote{Shamelessly ripped from \url{http://www.leeds.ac.uk/educol/abstract.htm}} describes what your abstract should be about.

The following text\footnote{Shamelessly ripped from \url{http://www.leeds.ac.uk/educol/abstract.htm}} describes what your abstract should be about.

The abstract should convey to the reader concisely and accurately within the space of a few sentences, the claim to knowledge that the authors are making. It should indicate the boundaries of space and time within which the enquiry has occurred. If there is a claim to generality beyond the boundaries of the enquiry the basis of that claim should be given, for example that a random sample is thought to be representative of a larger population. There should also be a hint of the method of enquiry.

The boundaries of an enquiry are important - and are unfortunately too often omitted from abstracts. This is due to the regrettable tendency for researchers to generalise their results from, for example, a few schools to all schools, and to imply that what was true at a particular time, is true for all time. Some reference to the geographical location of the children, or teachers, or schools on whom the claim to knowledge rests should be made. Because of the international nature of the research community it is worth making clear in what country the research took place. Also the period in which the data was collected should be stated.

The abstract should be a condensation of the substance of the paper, not a trailer, nor an introduction. Journals and thesis regulations usually put a limit of around 200 to 300 words to the length of an abstract. “Trailer” is a term borrowed from the cinema industry to describe a showing of a few highlights in order to win an audience. An “Introduction” tells that something is coming, but doesn’t reveal its substance. These are not what is needed.

Abstracts are recycled in abstract journals and electronic networks and provide the main vehicle for other researchers to become aware of particular studies. Hence the more clearly they convey the claim to knowledge of the original paper the more useful they are in helping the reader to decide whether it is worth taking the trouble to obtain and read the original and possibly cite it in his/her own writing.

Both the abstract and the paper should make sense without the other.

\clearpage

\tableofcontents
\mainmatter

% List of acronyms
\chapter*{Acronyms}
\begin{acronym}
    \setlength{\itemsep}{0.2em}
    \acro{ACR}{ACRONYM1}
    % etc.
\end{acronym}

% Include all your chapters as .tex files
\pagenumbering{arabic}

% introduction.tex

\chapter{Introduction}
\thispagestyle{chapterstart}

\section{Motivation}

The rapid advancement of quantum computing technology poses a significant threat to current cryptographic systems, particularly those based on integer factorization and discrete logarithm problems, such as RSA and \ac{ECC}. Quantum algorithms like Shor's algorithm can solve these problems efficiently, rendering classical cryptosystems insecure. To address this threat, the \ac{NIST} initiated a standardization process for \ac{PQC} algorithms that are secure against both quantum and classical attacks.

Among the winning candidates in this process is the lattice-based scheme CRYSTALS-Dilithium, known for its strong security foundations and efficient performance. However, while Dilithium is resistant to quantum attacks, its implementations on embedded devices are vulnerable to \acp{SCA}, which exploit physical leakages to extract sensitive information.

Side-channel resistance is critical for embedded devices, often with limited resources and physical accessibility. Enhancing the side-channel resistance of Dilithium implementations is essential for securing future cryptographic systems in the quantum era.

\section{Goals}

The primary goal of this paper is to analyze and compare various side-channel resistant implementations of the Dilithium digital signature scheme, focusing on methodologies, security levels, performance, and feasibility on embedded devices. Specifically, the objectives are:

\begin{itemize}
    \item \textbf{In-depth Analysis}: Examine selected implementations, highlighting techniques used for security and performance improvements.
    \item \textbf{Comparative Evaluation}: Compare implementations in terms of security levels (masking orders), performance metrics, and practical considerations.
    \item \textbf{Discussion of Trade-offs}: Identify and discuss trade-offs between security and performance in different masking techniques.
    \item \textbf{Insights for Deployment}: Offer strategies for deploying secure and efficient Dilithium implementations in resource-constrained environments.
\end{itemize}

By achieving these goals, this paper aims to contribute to understanding side-channel resistant implementations of Dilithium and guide future research.

\section{Structure of the Paper}

This paper is structured as follows:

\begin{itemize}
    \item \textbf{Chapter 2 - Background}: Provides an overview of post-quantum cryptography and quantum threats, introduces side-channel attacks, and discusses masking techniques relevant to Dilithium.
    \item \textbf{Chapter 3 - Analysis}: Presents an in-depth analysis of selected side-channel resistant implementations of Dilithium, examining methodologies and practical considerations.
    \item \textbf{Chapter 4 - Comparative Analysis}: Compares the analyzed implementations, highlighting strengths and weaknesses, discussing trade-offs, and implications for deployment.
    \item \textbf{Chapter 5 - Conclusion}: Summarizes the findings, discusses implications for deploying side-channel resistant Dilithium implementations, and suggests directions for future research.
\end{itemize}

By following this structure, the paper builds from foundational concepts to detailed analysis and comparison, culminating in conclusions that inform both academic and practical perspectives.

%\input{theory}
%\input{design}
%\input{implementation}
%\input{results}
\chapter{Conclusion}
\thispagestyle{chapterstart}

The secure deployment of post-quantum cryptographic (PQC) schemes like CRYSTALS-Dilithium on embedded devices is essential in the face of advancing quantum computing capabilities. However, practical implementations are susceptible to side-channel attacks due to inherent vulnerabilities, such as those in the bit-unpacking function and the Number Theoretic Transform (NTT).

In this paper, we analyzed the feasibility and performance of side-channel resistant implementations of Dilithium on embedded devices. We identified key vulnerabilities that can be exploited to recover secret keys and examined various countermeasures, including masking techniques and shuffling methods.

Our comparative analysis showed that high-order masking offers strong protection against side-channel attacks under rigorous security models like the $t$-probing model. However, such techniques often incur substantial performance overheads, particularly in non-optimized implementations. Optimizing these implementations through architecture-specific techniques, such as bitslicing, can mitigate performance penalties and make them more practical for resource-constrained environments.

Shuffling techniques and optimized masking methods provide a more favorable balance between security and efficiency. Shuffling introduces randomness to operation sequences, reducing predictability and side-channel leakage with lower performance impact. Optimized masking schemes, as demonstrated by Azouaoui et al. \cite{Azouaoui22}, enhance performance by focusing on critical operations and employing efficient gadgets.

Selecting appropriate countermeasures requires careful consideration of security requirements, performance constraints, and hardware characteristics. Practitioners should balance the desired level of side-channel resistance with acceptable performance overheads, tailoring implementations to their specific applications.

Future work should focus on developing hybrid approaches that combine masking with shuffling or bitslicing to achieve both high security and efficiency. Further optimization of implementations for specific architectures, mitigation of micro-architectural leakages, and exploration of alternative security models can enhance the practicality of side-channel resistant PQC schemes.

In conclusion, enhancing side-channel resistance is crucial for the secure adoption of PQC schemes in the quantum era. By carefully selecting and optimizing countermeasures, it is possible to implement practical and secure Dilithium on embedded devices. Ongoing research and development are essential to address the challenges of side-channel attacks and to ensure the robustness of cryptographic implementations against evolving threats.


% Generate list of figures
\newpage
\listoffigures

% Generate list of tables
\newpage
\listoftables
\clearpage

% Start back matter
\backmatter

% Generate bibliography with BibTeX
\bibliographystyle{alpha}
\bibliography{bibliography}

\end{document}
